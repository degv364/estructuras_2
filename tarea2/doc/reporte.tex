\documentclass {article}

\usepackage[spanish]{babel}
\usepackage [T1]{fontenc}
\usepackage [utf8]{inputenc}
\usepackage {graphicx}
\usepackage{subcaption}
\usepackage{listings}
\usepackage{multirow}

\begin {document}

\title{Tarea1: Simulación de cache}
\author{Daniel García Vaglio B42781, Esteban Zamora Alvarado B47769}

\maketitle


\section{Filtro Gaussiano}

El filtro gausiano es un filtro cuya respuesta al impulso es una funci\'o n gaussiana, esta est\'a
definida por \ref{eq:gauss_fn}. Estos filtros son utilizados para el an\'alisis de im\'agenes, cuando
se requiere comprimir im\'agenes, detectar bordes, o eliminar ruido. 

\begin{equation}
  \phi_{gauss}(x)=ae^{-\frac{(x-n)^2}{2x^2}}
\label{eq:gauss_fn}
\end{equation}

\section{Implementación}




\end{document}
