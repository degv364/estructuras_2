\documentclass {article}

\usepackage[spanish]{babel}
\usepackage [T1]{fontenc}
\usepackage [utf8]{inputenc}
\usepackage {graphicx}
\usepackage{color}
\usepackage{xcolor}
\usepackage{verbatim}
\usepackage{tabls}
\usepackage[space]{grffile}
\usepackage{url}
\usepackage{float}
\usepackage{listingsutf8}
\usepackage[justification=centering]{caption}
\usepackage{subcaption}
\usepackage{multirow}

\lstset{language=C++,
  frame=single,
  basicstyle=\ttfamily\small,
  keywordstyle=\color{blue}\ttfamily,
  stringstyle=\color{red}\ttfamily,
  commentstyle=\color[rgb]{0,0.5,0}\ttfamily,
  morecomment=[l][\color{magenta}]{\#},
  literate=%
  {á}{{\'a}}1
  {í}{{\'i}}1
  {é}{{\'e}}1
  {ó}{{\'o}}1
  {ú}{{\'u}}1
  {ñ}{{\~n}}1
}


\begin {document}

\title{Tarea1: Simulación de cache}
\author{Daniel García Vaglio B42781, Esteban Zamora Alvarado B47769}

\maketitle


\section{Filtro Gaussiano}

El filtro gaussiano es un filtro cuya respuesta al impulso es una función gaussiana, la cual está
definida por (\ref{eq:gauss_fn}). Dentro de las principales aplicaciones de los filtros gaussianos
se encuentra el procesamiento de imágenes con el fin de atenuar ruido. 

\begin{equation}
  \varphi_{gauss}(x)=ae^{-\frac{(x-n)^2}{2x^2}}
  \label{eq:gauss_fn}
\end{equation}

Estos filtros se suelen utilizar como una etapa de previa a los algoritmos de detección de bordes,
los cuales son muy susceptibles a detectar partículas de ruido como parte de los bordes en una
imagen. Por ejemplo, en la figura \ref{fig:canny} se puede ver el resultado de aplicar un algoritmo
de detección que utiliza un filtro gaussiano como preprocesamiento. En este se puede ver que al
modificar la desviación estándar del filtro ($\sigma$) se modifica el detalle de los bordes, lo cual
permite enfocarse en rasgos más o menos detallados según se requiera.

\begin{figure}[ht]
  \centering
  \includegraphics[width=\textwidth]{./img/canny}
  \caption{\label{fig:canny}Ejemplo de detección de bordes con diferentes parámetros \\del filtro gaussiano}
\end{figure}




\section{Implementación}

El programa adjunto computa imágenes filtradas, utilizando un filtro gaussiano, a partir de imágenes
fuente. Además realiza análisis automáticos del tiempo de ejecución y ``speedup'' para las
diferentes pruebas paralelizadas. Para cargar las imágenes se utilizan funcionalidades de
OpenCV. Cabe destacar que OpenCV solo se utiliza para cargar las imágenes y para mostrar los
resultados, no se utiliza para hacer el filtrado, sino que el mismo fue desarrollado enteramente por
el algoritmo implementado.

\subsection{Clase Image\_wrapper}
OpenCV para C++ ofrece una clase llamada Mat que carga imágenes de distintos formatos como una
matriz $n$ x $m$ (resolución en píxeles), en donde cada entrada es un vector tridimensional donde se
codifica el color (formato RGB). Para facilitar el manejo de esta clase, se crea un clase que
funciona como interfaz de Mat. Esta se llama \textit{Image\_wrapper}. Para accesar a una celda de la
matriz, OpenCV primero indexa la columna y luego la fila, esto no es el método natural, entonces
Image\_wrapper hace los ajustes necesarios para que la imagen se recorra con fila columna.


\subsection{Clase Neighborhood}
El filtro gaussiano opera haciendo cálculos en lo que se conoce como una ventana, la cual
corresponde a un conjunto de píxeles cercanos al píxel de interés, y usualmente tiene una forma
cuadrada. Para cada píxel se debe calcular un nuevo valor (el valor filtrado), aplicando la función
del filtro sobre dicha ventana o vecindario.

Se implementa una clase llamada \textit{neighborhood}, que consiste en una abstracción de la
ventana. Esta clase calcula el nuevo valor (valor filtrado) del píxel en su centro, tomando en
cuenta todos los píxeles dentro del ``vecindario'', además detecta cuando se encuentra cerca de los
bordes de la imagen y se ajusta para no accesar valores inválidos en sus cálculos (fuera de la
imagen).

\subsection{Algoritmo del filtro gaussiano}

Para calcular la imagen filtrada, se itera por cada uno de los píxeles (puede que en threads
distintos) y se calcula el nuevo valor dados los píxeles en el ``vecindario''. Se definen dos
imágenes, una es la imagen fuente o source, la cual es de solo lectura. Y está la imagen de
escritura, que es donde se guardan los valores nuevos. Esto se hace así para que conforme se van
cambiando los valores de los píxeles, estos cambios no afecten los nuevos cálculos, sino que los
valores filtrados se calculen todos con base en la imagen original. Sea $\sigma$ la desviación
estándar requerida, la función de Gauss bidimensional utilizada para el filtrado de la imagen se
define entonces por (\ref{eq:bidiGauss}).

\begin{equation}
  \varphi (x, y)= \frac{e^{-(x^2+y^2)/\sigma^2}}{2\pi \sigma^2} 
  \label{eq:bidiGauss}
\end{equation}

En la sección \ref{sec:pseudocodePixel}, se muestra el pseudocódigo del algoritmo utilizado para
obtener el nuevo valor en el píxel en la posición $(c_x, c_y)$, lo cual se repite para cada píxel de
la imagen. La ejecución de este algoritmo se reparte en los threads de ejecución, para poder
correrlo en paralelo, sin embargo, esta repartición se explica más adelante.


\subsubsection{Pseudocódigo para obtener nuevo valor en un píxel}
\label{sec:pseudocodePixel}

\begin{lstlisting}[escapeinside={<}{>}]
  //imagen fuente
  source
  //imagen de escritura
  target
  //vecindario centrado en (<$c_x, c_y$>)
  ng = source.neighborhood(<$c_x, c_y$>) 
  //valor nuevo por calcular
  target[<$c_x,c_y$>] = (0,0,0)

  for <$x$> in range(ng.x_start, ng.x_end):
     for <$y$> in range(ng.y_start, ng.y_end):
        <$d_x$> = <$x-c_x$> //distancia en x
        <$d_y$> = <$y-c_y$>
        gauss_val = <$\varphi(d_x, d_y)$> //calcular la función de gauss
        total+=gauss_val
        //calcular valores parciales con gauss
        target[<$c_x,c_y$>].red  += source[<$x,y$>].red*gauss_val 
        target[<$c_x,c_y$>].green += source[<$x,y$>].green*gauss_val
        target[<$c_x,c_y$>].blue += source[<$x,y$>].blue*gauss_val
  
  target[<$c_x,c_y$>].red   /= total //dividir entre el total
  target[<$c_x,c_y$>].green /= total
  target[<$c_x,c_y$>].blue  /= total
        
\end{lstlisting}

\subsection{Algoritmo paralelizado}
Para paralelizar el algoritmo del filtro gaussiano, se emplearon los \texttt{threads} de la
biblioteca estándar de C++ 11. De esta forma, se utiliza una función denominada
\texttt{gaussian\_filter} en cada thread, a la cual se le indica la región de la imagen sobre la
cual tiene que operar. En particular, la función fue implementada de forma que cada thread se
encarga de calcular el resultado del filtro para una franja vertical de la imagen, cuyo ancho
depende de la cantidad total de threads utilizada. Esto se puede observar en la figura
\ref{fig:image_threads}.


\begin{figure}[ht]
  \centering
  \includegraphics[width=0.6\textwidth]{img/image_threads}
  \caption{\label{fig:image_threads}Repartición de los datos de la imagen en los threads}
\end{figure}

En este sentido se puede afirmar que este algoritmo emplea paralelización a nivel de datos, ya que
se aplica la misma función para distintas secciones de la imagen. A continuación, se muestra el
segmento de código utilizado para implementar la repartición de los datos con los threads. Se puede
ver que los threads se recorren mediante un \texttt{for}, en cuyas iteraciones se va lanzando el
thread correspondiente, con la posición de la franja vertical indicada por el límite inferior
(\texttt{core\_id*interval}) y superior (\texttt{(core\_id+1)*interval}), en el eje x de la imagen.

\begin{lstlisting}[escapeinside={<}{>}]
  
  for (int core<\_>id=0; core<\_>id <<> num<\_>cores; core<\_>id++){
    //Ejecución del filtro gaussiano
    //A cada thread se le asigna una franja de la imagen
    threads[core<\_>id] = thread(<\&>gaussian<\_>filter,
    images[test],
    <\&>control,
    std<\_>dev,
    core<\_>id*interval,
    (core<\_>id+1)*interval);
  }
  //Se espera a que terminen de ejecutarse todos los threads
  for (int core<\_>id=0; core<\_>id < num<\_>cores; core<\_>id++){
    threads[core<\_>id].join();
  }

\end{lstlisting}


\subsection{Experimentation}

Se tiene un grupo de funciones, que se agrupan como funciones de experimentación. Estas se encargan
de computar, mostrar y analizar los resultados de las imágenes filtradas a partir de las imágenes
fuente. De ellas la más importante es la función \textit{experiment}, que es la que se encarga de
aplicar el algoritmo de filtrado en paralelo.


\section{Pruebas realizadas}
\label{sec:pruebas}
Para verificar el comportamiento del algoritmo implementado en secuencial y paralelo, se ejecutó la
función \texttt{experiment} múltiples veces con la misma imagen. Para cada ejecución de
\texttt{experiment}, se aplicó el filtro para cada cantidad de threads, la cual se incrementó en
cada prueba dentro de \texttt{experiment}, hasta cierta cantidad máxima de threads.

A partir de esto, se calcularon los tiempos de ejecución para cada prueba (aplicación del filtro),
con los cuales se obtuvieron los resultados de ``speedup'' con respecto a la ejecución secuencial
del algoritmo.

Por otro lado, solamente ejecutando el caso secuencial y el paralelizado con la cantidad máxima de
threads, se verificó que el resultado de ambos casos es el mismo, lo que permitió demostrar la
correctitud del algoritmo. Además, se visualizó el resultado del algoritmo paralelizado con respecto
a la imagen original, lo cual permitió observar que efectivamente se logra el efecto borroso
característico de este filtro.

\section{Resultados experimentales}

En la figura \ref{fig:result_filtro}, se puede ver la imagen original y las imágenes procesadas por
medio del filtro para una ventana de 9 píxeles y distintos valores de desviación estándar (figuras
\ref{fig:imag_proc1} y \ref{fig:imag_proc2}). En estas imágenes se puede notar el efecto esperado
del filtro gaussiano, en donde al incrementar la desviación estándar ($\sigma$) se aumenta la
dispersión del filtro, lo que incrementa el efecto borroso en la imagen.

En este ejemplo se puede ver como se disminuye considerablemente el ruido observado en la imagen
original, lo que demuestra la utilidad del filtro implementado. Cabe destacar, que las imágenes
mostradas corresponden a los resultados de la ejecución del algoritmo con 4 threads, sin embargo,
los resultados para la prueba secuencial son idénticos, ya que se comprobó que las matrices de la
imagen para ambos casos fueran iguales píxel por píxel.

\begin{figure}[H]
  \centering
  \subcaptionbox{Imagen original\label{fig:imag_orig}}{\includegraphics[width=0.3\textwidth]{img/image_original}}
  \subcaptionbox{Imagen procesada ($\sigma=1$)\label{fig:imag_proc1}}{\includegraphics[width=0.3\textwidth]{img/image_blurred_std_dev_1}}
  \subcaptionbox{Imagen procesada ($\sigma=5$)\label{fig:imag_proc2}}{\includegraphics[width=0.3\textwidth]{img/image_blurred_std_dev_5}}
  \caption{\label{fig:result_filtro}Resultados de la aplicación del filtro gaussiano implementado}
\end{figure}


Por otra parte, al aplicar la prueba indicada en la sección \ref{sec:pruebas} (para evaluar el
efecto del paralelismo), con una cantidad máxima de 15 threads, se obtuvo en primer lugar la gráfica
del tiempo de ejecución para cada cantidad de threads, la cual se puede observar en la figura
\ref{fig:time_multiple_15}.

En esta gráfica, cada línea de diferente color corresponde a una ejecución de la función
\texttt{experiment}, en donde se aplica toda la secuencia incremental de cantidad de cores. Es
importante destacar que en el hardware disponible en donde se ejecutaron las pruebas se tiene un
total de 4 cores en el procesador, lo cual se indica con la línea vertical punteada de color negro
en la gráfica. 

De esta manera, se puede notar que para todas las ejecuciones de \texttt{experiment} (diferentes
líneas), se tiende a tener una disminución en el tiempo de ejecución al incrementar la cantidad de
threads. 

Sin embargo, se puede notar que para todos casos, cuando se llega a 4 threads se obtiene el tiempo
mínimo alcanzable, ya que este no sigue disminuyendo al incrementar más la cantidad de threads. Esto
es consecuente con el hecho de que se tienen 4 cores físicos, por lo que incrementar los threads más
allá de este valor no hace más rápida la ejecución del programa, sino que incluso en algunos casos
el tiempo de ejecución aumenta.

\begin{figure}[H]
  \centering
  \includegraphics[width=0.9\textwidth]{img/time_multiple_15}
  \caption{\label{fig:time_multiple_15}Tiempo de ejecución para una cantidad variable de hasta \\15 threads (múltiples
    iteraciones de la prueba)}
\end{figure}

Por otro lado, es interesante notar que en la primera ejecución de \texttt{experiment} (línea
morada), se tiene un tiempo de ejecución mayor para los primeros valores de cantidad de threads con
respecto a las otras líneas. Esto se explica porque en esta primera ejecución se tienen por ejemplo
más misses de caché, ya que se requiere accesar a los datos por primera vez, mientras que en los
siguientes experimentos estos datos ya se encuentran cargados en el procesador.

En la figura \ref{fig:speedup_multiple_15}, se puede ver el ``speedup'' asociado a los tiempos de
ejecución de la gráfica anterior, con respecto al primer punto de la gráfica, que corresponde a la
ejecución secuencial del algoritmo. Se puede notar que el primer experimento (línea morada) tiene un
speedup considerablemente más grande que para los otros (líneas verde y cyan), lo cual se explica
porque, como se indicó previamente, la ejecución de la prueba secuencial en el primer experimento
tarda considerablemente más tiempo debido a que al principio se requiere traer los datos al caché
del procesador. Esto provoca que las siguientes pruebas (mayor cantidad de cores) para el primer
experimento tengan una mejora en tiempo mucho mayor, con respecto a la mejora observada en los otros
experimentos.

De cualquier forma, se puede ver que la ejecución paralelizada implementada del algoritmo del filtro
gaussiano, presenta una mejora en tiempo considerable con respecto a la ejecución secuencial, por lo
que se logró cumplir el objetivo deseado de aprovechar más el hardware disponible para hacer más
eficiente la ejecución del programa.

\begin{figure}[H]
  \centering
  \includegraphics[width=0.9\textwidth]{img/speedup_multiple_15}
  \caption{\label{fig:speedup_multiple_15}Speedup para una cantidad variable de hasta \\15 threads (múltiples iteraciones de la prueba)}
\end{figure}

En la figura \ref{fig:speedup_single_30}, se puede ver la misma prueba incrementando la cantidad de
threads pero hasta una cantidad máxima de 30 threads, ejecutando solamente una vez la función
\texttt{experiment} (una sola línea). En esta gráfica se puede ver con claridad que luego de 4
threads se llega al límite de la mejora real observada, ya que al seguir aumentando los threads, el
``speedup'' se mantiene alrededor del mismo valor, el cual incluso disminuye en algunos casos. 

\begin{figure}[H]
  \centering
  \includegraphics[width=0.9\textwidth]{img/speedup_single_30}
  \caption{\label{fig:speedup_single_30}Speedup para una cantidad variable de hasta \\30 threads (una iteración de la prueba)}
\end{figure}

\section{Conclusiones}
A partir del efecto observado sobre la imagen en la figura \ref{fig:result_filtro}, se puede afirmar
que se logró implementar correctamente el filtro gaussiano, logrando efectivamente corregir el ruido
de la imagen original de forma variable mediante los parámetros como la desviación estándar.

Por otro lado, de los resultados observados en las pruebas de ejecución del algoritmo del filtro con
una cantidad variable de threads, se puede afirmar que se cumplió el objetivo de disminuir el tiempo
de ejecución al incrementar la cantidad de threads, pero que la limitación del hardware disponible
no permitió observar una mejora mayor luego de 4 threads, ya que se tenían 4 cores físicos.

Además, es importante destacar que el primer experimento en la prueba realizada tarda un tiempo
mayor en la ejecución secuencial del algoritmo, debido a que se deben cargar los datos en el caché
del procesador, por lo que este primer resultado puede no reflejar el comportamiento promedio de
múltiples experimentos consecutivos.


\end{document}
