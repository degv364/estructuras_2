\documentclass {article}

\usepackage[spanish]{babel}
\usepackage [T1]{fontenc}
\usepackage [utf8]{inputenc}
\usepackage {graphicx}
\usepackage{color}
\usepackage{xcolor}
\usepackage{verbatim}
\usepackage{tabls}
\usepackage[space]{grffile}
\usepackage{url}
\usepackage{float}
\usepackage{listingsutf8}
\usepackage[justification=centering]{caption}
\usepackage{subcaption}
\usepackage{multirow}
\usepackage{apacite}

\lstset{language=C++,
  breaklines=true,
  frame=single,
  basicstyle=\ttfamily\small,
  keywordstyle=\color{blue}\ttfamily,
  stringstyle=\color{red}\ttfamily,
  commentstyle=\color[rgb]{0,0.5,0}\ttfamily,
  morecomment=[l][\color{magenta}]{\#},
  literate=%
  {á}{{\'a}}1
  {í}{{\'i}}1
  {é}{{\'e}}1
  {ó}{{\'o}}1
  {ú}{{\'u}}1
  {ñ}{{\~n}}1
}


\begin {document}

\title{Tarea 3: Uso de Gem5}
\author{Daniel García Vaglio (B42781), Esteban Zamora Alvarado (B47769)}

\maketitle

\section{¿Qué es Gem5?}
La herramienta Gem5 consiste en una plataforma de software libre (licencia BSD) para la simulación
de sistemas computacionales basada en la ejecución de una serie de eventos discretos
\cite{binkert2017}, los cuales conforman la simulación temporal del sistema. Al ser modular y
abierto, Gem5 permite modificar, extender o reemplazar sus módulos para adaptarse a las necesidades
del usuario, mediante la exposición de la clase \texttt{SimObject}, con la cual se pueden crear
nuevos objetos para agregarlos al sistema de simulación.

Esta plataforma está orientada a la investigación en arquitecturas de computadoras, por lo que tiene
aplicaciones a nivel académico y en la industria, soportando arquitecturas como MIPS, ARM, Power,
SPARC y x86 \cite{binkert2017}. Con respecto al software de Gem5, el mismo está implementado
principalmente en C++ y Python, en donde el usuario escribe un script de configuración en Python
para establecer las condiciones de la simulación, las características del hardware simulado, el
modelo de memoria y procesamiento, entre otros.

Más específicamente, la ejecución de Gem5 utiliza el script de configuración establecido por el
usuario para correr cierto binario objetivo compilado para la arquitectura de interés, presentando
mediante la salida estándar cierta información sobre la simulación realizada y la salida propia del
binario objetivo, y mediante una serie de archivos, la lista de objetos de simulación creados y las
estadísticas de la simulación (tiempo de ejecución, uso de memoria, entre otros).

\subsection{Modos de simulación en Gem5}
Dentro de los modos de simulación disponibles en Gem5 se encuentra ``Full System Mode'', en el cual
se simula el sistema de hardware completo, lo cual comprende el CPU, la memoria, dispositivos de
entrada y salida (IO), entre otros; de forma que esta herramienta se comporta como un hypervisor
\cite{lowe2017}. Este modo permite entonces exponer los efectos de más bajo nivel sobre el
rendimiento del sistema computacional, asociados a la interacción del sistema operativo con el
hardware, de manera que el usuario incluso debe brindar una versión del kernel de Linux para
ejecutarlo. Por otra parte, en ``System-call Emulation Mode'' se simula el programa binario objetivo
y la respuesta del sistema operativo a dicho programa mediante llamadas de sistema, por lo que no se
simula el efecto del hardware subyacente.

\subsection{Modelos de CPU en Gem5}
Gem5 permite utilizar diferentes modelos de CPU para ejecutar las simulaciones, los cuales
comprenden diferentes niveles de complejidad a nivel arquitectónico (modelo del pipeline,
interacción con memoria, entre otros).

En primer lugar, el modelo \texttt{SimpleCPU} es uno puramente funcional con ejecución en orden, el
cual es adecuado cuando no se requiere simular un modelo más complejo. Más detalladamente, este
modelo mantiene el estado arquitectónico del programa mediante registros, permite ejecutar rutinas
de interrupción, hacer solicitudes (``fetch'') a memoria y actualizar el PC (``Program Counter'') al
avanzar en la ejecución de las instrucciones \cite{binkert2017}. Además \texttt{SimpleCPU} permite
dos submodos, \texttt{AtomicSimpleCPU}, el cual emplea accesos atómicos a memoria (solicitando y
obteniendo los datos en una misma transacción), y el \texttt{TimingSimpleCPU}, en el cual se
realizar accesos a memoria temporizados (simula de forma detallada las latencias de acceso a la
jerarquía de memoria).

Por otra parte, el modelo \texttt{O3CPU} corresponde al más detallado de Gem5, ya que implementa un
Pipeline con ejecución fuera de orden (``Out of Order Execution''), el cual tiene las etapas de
\textit{Fetch}, en donde en cada ciclo se trae una nueva instrucción al Pipeline desde la jerarquía
de memoria y se gestiona el ``Branch Prediction'', \textit{Decode}, en la cual se interpreta la
instrucción traída al Pipeline y se resuelve el valor del PC en los branches condicionales,
\textit{Rename}, en donde se realiza el ``Register Renaming'' para prevenir algunos riesgos de
datos, \textit{Issue/Execute/Writeback}, que corresponde al conjunto de etapas en donde se simula el
envío de las instrucciones al ``Instruction Queue'', se gestiona el envío a ejecución de las
instrucciones disponibles de este ``Instruction Queue'' (\textit{Issue}), se lleva a cabo la
ejecución fuera de orden de estas instrucciones (\textit{Execute}) y se reescriben los resultados en
los registros correspondientes (\textit{Writeback}), y \textit{Commit}, en donde se manejan las
posibles excepciones provocadas por las instrucciones y se concreta el estado arquitectónico del
sistema. Para lograr esto, este modelo contiene recursos como ``Branch Predictor'', ``Reorder
Buffer'', ``Instruction Queue'', ``Load-store Queue'' y unidades funcionales de ejecución
\cite{binkert2017}.

También, existen otros modelos de CPU como \texttt{InOrderCPU}, el cual implementa un Pipeline con
ejecución en orden con etapas similares a las del modelo \texttt{O3CPU}, y el modelo
\texttt{TraceCPU}, el cual permite rastrear la información de timing y dependencias del sistema
computacional sin requerir de la utilización del modelo detallado pero más lento de \texttt{O3CPU},
lo que permite explorar el rendimiento del sistema de forma más rápida con una precisión razonable
\cite{binkert2017}.


\section{Escenarios de prueba}

\subsection{Instalación}
Como se indica en las instrucciones, para instalar el simulador se emplea una máquina virtual de
Virtual Box, en Ubuntu 16.04 LTS. En la figura \ref{fig:specs_vbox}
se muestran las especificaciones utilizadas par la realización de la tarea.

\begin{figure}[H]
  \centering
  \includegraphics[width=0.5\textwidth]{img/virtualbox_spec.png}
  \caption{\label{fig:specs_vbox} Especificaciones de la máquina virtual}
\end{figure}
 
Luego para instalar Gem5, lo primero que se debe hacer es instalar las dependencias. Como muchas de
ellas se encuentran en los repositorios denominados “Universe”, es necesario verificar que esos
repositorios están agregados. En caso que no esté se debe agregar, en nuestro caso particular, sí estaba. Luego se instalan las dependencias:

\begin{lstlisting}
  cat /etc/apt/sources.list
  sudo apt-get install g++ python python-dev scons zlib1g-dev m4  python-pydot protobuf-compiler git
\end{lstlisting}
 
Luego es necesario clonar el repositorio (ver figura \ref{fig:cloning}). Luego se debe compilar el
código fuente. En la figura \ref{fig:compiling} se muestra un screenshot de la compilación. Además
se muestra una comparación de la ejecución de procesos de la máquina virtual y del host (utilizando
htop). Mientras la máquina virtual compila(terminales de abajo), el host únicamente ve los procesos
de virtualbox (terminales de arriba). 

\begin{lstlisting}
  mkdir ~/repos && cd ~/repos
  git clone https://github.com/gem5/gem5.git
  cd ~/repos/gem5
  scons build/ARM/gem5.opt -j5
\end{lstlisting}



\begin{figure}[H]
  \centering
  \includegraphics[width=\textwidth]{img/cloning.png}
  \caption{\label{fig:cloning} Clonando Gem5}
\end{figure}

\begin{figure}[H]
  \centering
  \includegraphics[width=\textwidth]{img/compilin7_extern.png}
  \caption{\label{fig:compiling} Compilación de Gem5 y comparación con el host}
\end{figure}

Como se deben realizar simulaciones con un sistema completo, entonces se deben descargar las
imágenes que recomiendan los desarrolladores de Gem5. Para los siguientes comandos se supone que el
archivo con las imágenes ha sido descargado en ``\~/Downloads/'' (ver figura \ref{fig:extracted}). 

\begin{lstlisting}
  sudo mkdir /dist/m5/system
  cd /dist/m5/system
  sudo cp ~/Downloads/aarch-system-2014-10.tar.xz .
  sudo tar -xf aarch-system-2014-10.tar.xz
\end{lstlisting}

\begin{figure}[H]
  \centering
  \includegraphics[width=0.5\textwidth]{img/extracted_img_arm.png}
  \caption{\label{fig:extracted} Imágenes de sistema completo recomendadas por Gem5}
\end{figure}

Una vez completados los pasos anteriores, el sistema está listo para correr simulaciones con Gem5.

\subsection{Escenario A}

\subsubsection{Emulación de Syscall}
Primero se trabaja con el modo de simulación de llamadas a sistema. Para esto se debe ejecutar gem5,
indicarle que se va a utilizar dicha configuración, y además se le proporciona el ejecutable que se
desea ejecutar en la simulación. Para este caso particular, se utiliza el mismo ``hello world'' que se
ofrece de ejemplo simple de prueba. Para eso se utiliza el siguiente comando:

%FIXME: break line
\begin{lstlisting}
  build/ARM/gem5.opt configs/example/se.py -c tests/test-progs/hello/bin/arm/linux/hello
\end{lstlisting}

El código de hello world se ofrece a continuación:

\begin{lstlisting}
  #include <stdio.h>
  int main(){
    printf("Hello world!");
  }
\end{lstlisting}

El resultado de la simulación se muestra en la figura %ref

\begin{figure}[H]
  \centering
  \includegraphics[width=\textwidth]{img/hello_se_full_frame.png}
  \caption{\label{fig:se_arm} Simulación SE de hello world en Gem5}
\end{figure}

\subsubsection{Simulación de sistema completo}
El primer paso es crear la imagen que se va a estar utilizando para la simulación. Se toma como base
una de las imágenes disponibles de los links de descarga: ``linux-aarch32-ael.img''. Se crea una
copia y se monta para poder editar la imagen. Cabe destacar que como esto es una imagen de un disco
completo, se debe indicar el inicio del mismo al montarlo.

\begin{lstlisting}
  cd /dist/m5/system
  cp linux-aarch32-ael.img my_image.img
  fsik -l my_image.img
  #start: 63, block size 512
  sudo mount -o loop,offset=32256 my_image.img /mnt
\end{lstlisting}

Una vez se tiene montada la imagen, se le agrega el ejecutable y se desmonta:
\begin{lstlisting}
  cd /mnt/bin
  sudp cp ~/repos/gem5/test/test-progs/hello/bin/arm/linux/hello .
  cd
  sudo umount /mnt
\end{lstlisting}

Para la simulación del sistema completo se utiliza la imagen que se creó anteriormente.
Las simulaciones de sistema completo no imprimen en la salida estándar, sino que se
debe realizar una conexión por telnet con la máquina simulada. Entonces primero, para poder simular
la máquina se debe ejecutar el siguiente comando, desde la raíz de Gem5 (ver figura
\ref{fig:arm_full_gem5}):

\begin{lstlisting}
  build/ARM/gem5.opt configs/example/fs.py --disk-image=/dist/m5/system/disk/my_image.img
\end{lstlisting}

\begin{figure}[H]
  \centering
  \includegraphics[width=\textwidth]{img/full_sys_arm.png}
  \caption{\label{fig:arm_full_gem5} Simulación FS en Gem5}
\end{figure}

Luego en otra terminal se ejecuta telnet para conectarse a ``localhost 3456'' (ver fig
\ref{fig:telnet}). Una vez realizada la
conexión, se debe esperar a que el sistema bootée y luego se logea como root. Una vez dentro, como
el ejecutable fue agregado dentro de /bin, basta con digitar ``hello'' para poder ejecutarlo (ver
figura \ref{fig:hello_fs}).

\begin{figure}[H]
  \centering
  \includegraphics[width=\textwidth]{img/full_telnet_conn.png}
  \caption{\label{fig:telnet} Conexión por telnet}
\end{figure}

\begin{figure}[H]
  \centering
  \includegraphics[width=\textwidth]{img/hello_fs_full_frame.png}
  \caption{\label{fig:hello_fs} Ejecución en Full System de Hello}
\end{figure}

\subsection{Escenario B}

\subsection{Escenario C}

Para realizar la simulación se esta parte se utiliza la configuración básica de llamadas de
sistema. Para ejecutar la prueba se utiliza el siguiente comando (ver figura \ref{fig:exec_for_viz}):

\begin{lstlisting}
  ./build/ARM/gem5.opt --debug-flags=O3PipeView --debug-start=0 --debug-file=trace.out configs/example/se.py --cpu-type=O3_ARM_v7a_3 --caches -c tests/test-progs/hello/bin/arm/linux/hello
\end{lstlisting}

La primera parte indica el ejecutable de Gem5 que se utiliza. En este caso se refiere a la versión
par ARM. Luego se le indica que se quiere generar la salida para el visualizador O3 Pipe View. La
tercera opción se utiliza para indicar el inicio. Como en este caso se quiere analizar toda la
simulación se coloca 0 (desde el tick 0 del clock). Posteriormente se indica el archivo donde se
deben guardar los resultados. Finalmente se proporciona el script de configuración con sus
respectivas opciones. Luego se debe transformar el archivo de salida a un archivo en texto con
colores, con el siguiente comando:

\begin{lstlisting}
  ./util/o3-pipeview.py -c 500 -o pipeview.out --color m5out/trace.out
\end{lstlisting}

Las visualizaciones se proporcionan en la figura \ref{fig:result_viz}. La primera parte es la
inicialización (ver figura \ref{fig:pipeline_init}). Como era de esperarse hay muchos misses, esto
porque la información no se ha cargado en caché. Luego en la figura \ref{fig:pipeline_waw} se tiene
un caso de ``write after write'', entonces las últimas instrucciones experimentan un stall. En la
figura \ref{fig:viz} se muestra la operación normal, este tipo de patrones se repite varias veces
en el archivo. Luego en la figura \ref{fig:viz2} hay un branch, es importante notar que estas
instrucciones usualmente provocan un cambio entre ``='' y ``.'' . Además también se sabe que no se
toma el branch, pues la siguiente instrucción se encuentra 4 posiciones luego de la
pasada. Finalmente en la figura \ref{fig:viz3}, hay un salto. Este no es un salto directamente, sino
que se cambia el valor de PC. Como es de esperarse, las instrucciones siguientes tienen misses al
hacer fetch de las instrucciones, esto porque las instrucciones aún no están en cache.

\begin{figure}[ht]
  \centering
  \includegraphics[width=\textwidth]{img/for_view_full_frame.png}
  \caption{\label{fig:exec_for_viz} Ejecución de Gem5 para visualización}
\end{figure}

\begin{figure}[h]
  \centering
  \label{fig:result_viz}
  \begin{subfigure}[b]{0.9\textwidth}
    \includegraphics[width=\textwidth]{img/pipeline_init.png}
    \caption{\label{fig:pipeline_init} Inicialización}
  \end{subfigure}
  
  \begin{subfigure}[b]{0.9\textwidth}
    \includegraphics[width=\textwidth]{img/pipeline_waw.png}
    \caption{\label{fig:pipeline_waw} WAW}
  \end{subfigure}
  \hfill
  
  \begin{subfigure}[b]{0.9\textwidth}
    \includegraphics[width=\textwidth]{img/exec_nice.png}
    \caption{\label{fig:viz} Ejecución normal}
  \end{subfigure}
  
  \begin{subfigure}[b]{0.9\textwidth}
    \includegraphics[width=\textwidth]{img/exec_nice2.png}
    \caption{\label{fig:viz2} Ejecución con branch}
  \end{subfigure}
  
  \begin{subfigure}[b]{0.9\textwidth}
    \includegraphics[width=\textwidth]{img/exec_nice3.png}
    \caption{\label{fig:viz3} Misses de fetch}
  \end{subfigure}
  \caption{Visualización del pipeline}
\end{figure}





\section{Conclusiones}


%----------------------
% Bibliografía
%----------------------

\bibliographystyle{apacite}
\bibliography{bibliografia}

\end{document}
