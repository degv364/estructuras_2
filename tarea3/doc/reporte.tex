\documentclass {article}

\usepackage[spanish]{babel}
\usepackage [T1]{fontenc}
\usepackage [utf8]{inputenc}
\usepackage {graphicx}
\usepackage{color}
\usepackage{xcolor}
\usepackage{verbatim}
\usepackage{tabls}
\usepackage[space]{grffile}
\usepackage{url}
\usepackage{float}
\usepackage{listingsutf8}
\usepackage[justification=centering]{caption}
\usepackage{subcaption}
\usepackage{multirow}
\usepackage{apacite}

\lstset{language=C++,
  frame=single,
  basicstyle=\ttfamily\small,
  keywordstyle=\color{blue}\ttfamily,
  stringstyle=\color{red}\ttfamily,
  commentstyle=\color[rgb]{0,0.5,0}\ttfamily,
  morecomment=[l][\color{magenta}]{\#},
  literate=%
  {á}{{\'a}}1
  {í}{{\'i}}1
  {é}{{\'e}}1
  {ó}{{\'o}}1
  {ú}{{\'u}}1
  {ñ}{{\~n}}1
}


\begin {document}

\title{Tarea 3: Uso de Gem5}
\author{Daniel García Vaglio (B42781), Esteban Zamora Alvarado (B47769)}

\maketitle

\section{¿Qué es Gem5?}
\cite{opencv} %avoid compilation errors


\section{Escenarios de prueba}



\section{Filtro Gaussiano}





\section{Conclusiones}


%----------------------
% Bibliografía
%----------------------

\bibliographystyle{apacite}
\bibliography{bibliografia}

\end{document}
